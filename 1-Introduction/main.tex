\documentclass[main.tex]{subfiles}

\begin{document}
\pagenumbering{arabic}	% Numbering style for body

\chapter{Introduction}
\chaplabel{introduction}
\section{Background and motivation}
\seclabel{background}

It is estimated that over one hundred million landmines are currently unaccounted for in past and present conflict zones across the globe. Remaining active for many years after burial, these indiscriminate weapons are responsible for injuring thousands of civilians and military personnel each year \parencite{landmineMonitor2015}.

The Defence Science and Technology Group (DSTG) is an Australian Government organisation that applies scientific methods and technologies with the aim of defending Australia's national interests. A current focus for the DSTG is reducing the risks posed to Australian Defence Force (ADF) personnel by landmines in conflict zones. There are two primary methods of landmine detection currently employed by the ADF: manual screening with the use of a handheld metal detector, and vehicle mounted screening using ground penetrating radar (GPR) arrays.
Both methods require personnel to be in close proximity to landmines due to the need for manual control of the sensors. Another major drawback to both methods is the difficulty to distinguish between landmine and non-landmine objects, which results in a large number of false positives - where a non-landmine object is incorrectly identified as being a mine. This leads to demining operations being very time consuming as extreme care must be taken with each potential threat. The inefficiency and high risk associated with these demining operations is the major motivation for the current project. 

As a solution, the DSTG is investigating systems that can locate and identify a range of threats with increased speed and accuracy, while at the same time removing operators from potential harm. 
This project proposes the development of an autonomous and unmanned platform to mount landmine detection equipment, which would eliminate the risk to operators entirely. Through the use of a multi-sensor system consisting of a metal detector and a GPR, combined with sensor fusion, the aim will be to reduce false positives thereby increasing detection accuracy.

\nomenclature[A]{DSTG}{Defence Science and Technology Group}% 
\nomenclature[A]{ADF}{Australian Defence Force}% 
\nomenclature[A]{GPR}{Ground Penetrating Radar}%

\section{Project goal}
\seclabel{projectGoal}

The goal of this project is to develop an autonomous platform for landmine detection, with a focus on the reduction of false positives. The project objectives necessary to achieve this goal are described below.
 
\subsection{Primary objectives}
\seclabel{primary}
The primary objectives are the minimum allowable outcomes for the project and should be delivered upon completion. The objectives are derived from the project goal, and successful completion of these objectives will also allow the goal to be achieved. These objectives are:
\begin{enumerate}
\item \textbf{Selection and subsequent modification of an existing platform for control via a remote device}\\ Necessary systems for control, such as throttle, steering, gear and brake actuation, and communication will be implemented on the platform. Each system will be tested separately to ensure that it functions correctly, meets the project requirements, and is safe to operate remotely. Once all systems are operating independently, simultaneous operation shall be conducted with the aim to maintain controllability of the platform at all instances. Successful completion of this objective is the first step towards full platform automation.  

\item \textbf{Software development for platform automation and basic navigation, allowing for the platform to travel autonomously under supervision from an operator}\\ Initially, automation and navigation software will be tested in a virtual environment, allowing for full simulation of pre-defined scenarios in a risk-free manner. The outputs of this virtual platform will include the location, heading, and speed of the platform. On completion of objective 1, the software and control systems will be integrated into the real platform, and further testing will be performed. After completing this objective, the platform will be fully autonomous.

\item \textbf{Automatic traversal of a user defined path or region}\\ In real world scenarios, it is important for the operator to be able to specify the path of the platform as they wish. Initially, single waypoint navigation will be implemented, allowing an operator to instruct the platform to travel from one predefined point to another before coming to a stop. Once this is achieved, the operator will be able to define waypoints that enclose a region. The platform will then subdivide the region and generate its own path to travel. Software will again be tested in a virtual environment before being implemented into the real platform. Completion of this objective will extend the navigational capabilities of the platform, and make it easier for an operator to define a mission. 

\item \textbf{Modification of an existing platform to accommodate a metal detector and a GPR}\\ A sensor mount will be designed and built so that it can attach to the platform and support the chosen sensors safely, while adhering to the operating requirements of both the platform and sensors. The most suitable design will be chosen from several concepts based on its ability to isolate the sensor suite and minimise sources of interference to the sensor output. On completion of this objective, the platform will be fully integrated with the sensor suite.

\item \textbf{Software development for detection and identification of subsurface objects, with a focus on reducing the number of incorrect identifications or false positives}\\ After initial testing, the output signals from the metal detector and GPR will be processed. The results will be analysed with the aim of identifying suitable metrics that can be used to classify and differentiate various metallic and non-metallic targets. A database of metrics will be produced and used to identify the likelihood of a landmine being present. Operational trials will be conducted to test the effectiveness of the developed system. On completion of this objective, the sensor suite will be able to detect targets.
\end{enumerate}

\subsection{Extended objectives}
\seclabel{extendedObjectives}
Extended objectives are those that are beyond the scope of the project, but may be attempted on completion of the primary objectives. While these are not necessary to achieve the project goal, they add useful functionality. These objectives are:

\begin{enumerate}
\item \textbf{Development of a tablet application that allows for communication and control of the platform and sensor systems}\\ Rather than receiving information on a computer, a portable handheld device such as a tablet can be used. A dedicated application will be developed, allowing for even an unskilled operator to carry out a complex mission. The operator will be able to control and monitor the progress of the platform in real time, and once an object is detected by the sensor system, the operator will also be notified. A GPS coordinate will be assigned to the object, and based on the returned sensor data, the operator can make an informed decision about how to proceed.

\item \textbf{Provision of a live video feed to the operator}\\ A live video feed allows for an operator to have a better understanding of the operating environment. A camera will be installed on the platform, and the output will be integrated with the developed tablet application. This will allow the operator to detect and respond appropriately to unknown obstacles in the search area. 

\item \textbf{Physical marking of landmine locations by the platform at the time of detection}\\ GPS coordinates alone may not provide sufficient precision for identifying the location of landmines. An electromechanical system that physically marks the location would allow demining personnel to identify this location more clearly. On completion of primary objective 5, the marking system will be integrated with the platform and tested in operational trials. 

\end{enumerate}

\section{Scenario of operation}
\seclabel{SoO}
This section will look at real world environments that the platform could be expected to operate in. In addition, the mission profile, which is the intended use case for the platform, is presented. Based on these, a set of performance specifications are developed.  

\subsection{Operating environments}
\seclabel{operatingEnvironments}
There are a large number of countries in which ADF personnel are deployed, each of which has differing terrain, vegetation, soil types, landmine types and sizes of landmine affected areas. Since it is impossible to develop a platform that suits all real world operating conditions, the Middle Eastern region is selected as the operating environment, since this is the location to which the largest number of ADF personnel are deployed. 

Middle Eastern regions are heavily effected by landmines and unexploded ordinance due to many years of conflicts. Landmines have been left in over 11 countries from internal civil wars or from external battles with neighbouring countries. There is an estimated 23 million landmines buried throughout this region with landmine technologies and materials ranging back to World War II. The primary goal of burying landmines is to conceal their location, therefore they are typically buried from 5 cm to 15 cm below the surface, depending on the time available to bury them.

Terrain properties for these regions are predominantly covered with small granular sands that have little to no nutritional value for vegetation.  

two such environments are considered, both of which are heavily affected by landmines \parencite{AustralianGovernment2016} \textcolor{red}{Maz: describe it in general terms first, then move onto case studies}. 

\subsubsection{Eqypt}
\seclabel{Egypt}


Egypt is listed as the country most contaminated by landmines, with an area of roughly 25,000 square kilometres containing over 23 million buried anti-tank and anti-personnel mines \parencite{Rushfan2008}. The majority of these mines were placed during World War II and consist of high metal content anti-tank and anti-personnel mines \parencite{Khamis13}. The main contaminated areas are along the North coast and the Sinai Peninsula. These areas represent 22\% of the total area of Egypt and consist predominantly of lithosols with large flat expanses of sandy land \parencite{Nahrawy2011}. Temperatures in excess of 50 degrees are common, making it difficult for demining personnel to operate for extended periods of time \parencite{Khamis13}. 
%Over time, shifting sands allow these landmines to sink deeper to depths of up to 2 metres, beyond the range of most landmine detection techniques \parencite{Khamis13}. 
%The chosen platform would then be required to travel over sandy terrains and over small obstacles such as rocks, with the sensors being able to detect landmines through sandy and loose soil to depth of up to 15 cm. 

 \subsubsection{Iran}
 \seclabel{Iran}
Iran has an area of approximately 18,000 square kilometres of land contaminated with 12-16 million landmines left over from the 1980-1988 Iran-Iraq conflict \parencite{landmineMonitor2015}. The main areas of contamination are western provinces such as Khuzestan, which has mountainous terrain and sandy soils \parencite{Aman16}. 
%Both anti-tank and anti-personnel mines can be found, with shifting sands causing landmines to sink to depths greater than 80 cm \parencite{Aman16}.

\textcolor{red}{This paragraph and the one before it need more information about terrain, vegetation, soil type, moisture, mine type, \textbf{mine depths}, obstacles, gradient, whatever else can be found! These should then lead provide the basis for the details presented at the start of the mission profile. Or maybe have an additional subsection, which describes the chosen operating environment... - RK}

\subsection{Mission profile}
\seclabel{MissionProfile}
A typical demining operation requires personnel to clear a path or area, ensuring it is safe for ground troops and vehicles to pass over. The mission profile developed for the platform resembles such an operation, and will form the basis of many of the project's design requirements.\textcolor{red}{Relate more explicitly to the case studies from previous section}
\begin{quote}The mission takes place in a region with dry, sandy soils and level terrain unobstructed by vegetation or obstacles, such as a road or open field. A number of types of landmines, including high and low metal content anti-personnel and anti-tank mines, can be found buried at depths of up to 15 cm. The operator will use a remote device to select waypoints that define a path or the boundaries of an area of interest. The platform will travel out to this path or area, then autonomously navigate it. Using the onboard sensor suite, the platform scans the area in real time at an operational speed of 5 km/h. Whenever a detection is made, the platform will stop and log the GPS coordinates of the location, with a positional error of less than 0.5 m. Sensor data and a confidence interval are then sent to the operator, giving the likelihood of the detected object being a landmine. Once the operator instructs the platform to resume, it continues to scan the area. Over time, a map of the scanned region showing all detected objects is developed using the sensor data. At any point the operator can abort the mission, or take manual control of the platform.
\end{quote}
\textcolor{red}{Add pictures of schematics}

\subsection{Performance specifications}
\seclabel{performance}
Knowing the basic operating environment and mission profile, performance specifications can be defined, encompassing the platform and sensor systems. \textcolor{red}{In this section, explain more clearly how each performance specification satisfies the mission profile, and relates to the operating environment}

\subsubsection{Platform}
\seclabel{platform}
In order to be able to autonomously navigate a region of interest and meet the objectives outlined earlier, the platform must meet the following requirements: 
\begin{itemize}
 \item Carry a total payload of 100 kg. A pair of metal detector and GPR arrays with their respective control boxes have an estimated gross weight of 80 kg. This allows 20 kg for any mounting structures,  electronics and computers. 
 \item Travel off-road in regions with dry sandy soils and level unobstructed terrain.
\item Travel in a straight line, both forwards and reverse, and perform turns while maintaining an operational speed of 5 km/h. \textcolor{red}{Relate this to the operating environment, or add these details there}
\item Be manoeuvrable enough to navigate the region of interest, yet controllable enough that in case of an emergency, it can come to a complete stop from its operational speed in 1 m.
\end{itemize}

\subsubsection{Sensors}
\seclabel{sensors} 
The requirements for the sensor system are as listed below:
\begin{itemize}
\item Detect high and low metal content anti-personnel and anti-tank mines. \textcolor{red}{More details}
\item Identify mines at depths of up to 15 cm at an operational speed of 5 km/h.
\item Discriminate between different landmines and other objects, aiming to minimise the number of false positives detected.
\end{itemize}

\section{Project scope}
\seclabel{projectScope}

%Acceptance criteria: The conditions that must be met before project deliverables are accepted.
%Deliverables: The products, services, and/or results your project will produce (also referred to as objectives).
%Project Exclusions: Statements about what the project will not accomplish or produce.
%Constraints: Restrictions that limit what you can achieve, how and when you can achieve it, and how much achieving it can cost.
%Assumptions: Statements about how you will address uncertain information as you conceive, plan, and perform your project.

The aim of this project is to demonstrate the feasibility of a concept that incorporates an autonomous platform and a real time multisensor landmine detection system. It is not intended for the final design solution to be a commercially viable device that is ready to be deployed for the ADF. As of such, 

\textcolor{red}{\textbf{Need to add more things here, this is just the start.} Things that are outside of scope: objection avoidance, mine clearance, }
\end{document}