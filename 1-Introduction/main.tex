\documentclass[main.tex]{subfiles}

\begin{document}
\pagenumbering{arabic}	% Numbering style for body

\chapter{Introduction}
\chaplabel{introduction}
\section{Background and motivation}
\seclabel{background}

It is estimated that over one hundred million landmines are currently unaccounted for in past and present conflict zones across the globe. Remaining active for many years after initial deployment, these indiscriminate weapons are responsible for injuring thousands of civilians and military personnel each year \parencite{landmineMonitor2015}.

The Defence Science and Technology Group (DSTG) is an Australian Government organisation that applies scientific methods and technology with the aim of defending Australia and its national interests. A current focus for the DSTG is reducing the risks posed to Australian Defence Force (ADF) personnel by landmines in conflict zones. There are two primary methods of landmine detection currently employed by the ADF: manual screening with the use of a handheld metal detector, and vehicle mounted screening using both metal detector and ground penetrating radar (GPR) arrays. Both methods place operators in close proximity to landmines, while being unable to consistently identify them, which results in a large number of incorrect detections or false positives. The inefficiency and high risk associated  with these demining operations is the major motivation for the current project. 

As a solution, the DSTG is investigating systems that can locate and identify a range of threats with increased speed and accuracy, while at the same time removing operators from potential harm. 
This project proposes the development of an autonomous and unmanned platform to mount landmine detection equipment, which would eliminate the risk to operators entirely. Through the use of an integrated multi-sensor system consisting of a metal detector and a GPR, the aim will be to reduce false positives, thereby increasing detection accuracy.

\nomenclature[A]{DSTG}{Defence Science and Technology Group}% 
\nomenclature[A]{ADF}{Australian Defence Force}% 
\nomenclature[A]{GPR}{Ground Penetrating Radar}%

\section{Project goal}
\seclabel{projectGoal}

The goal of this project is to develop an autonomous platform for landmine detection, with a focus on the reduction of false positives. Achieving this goal will require multi-sensor integration and signal processing, as well as software development for platform navigation and automation. The project objectives necessary to achieve this goal are described below.
 
\subsection{Primary objectives}
\seclabel{primary}
The primary objectives are the minimum allowable outcomes for the project and should be delivered upon completion. These objectives are:
\begin{enumerate}
\item Selection and subsequent modification of an existing platform for remote control via a handheld device. Necessary systems for control, such as throttle, steering, gear and brake actuation, and communication will be implemented on the platform. Each system will be tested separately to ensure that it functions correctly and is safe to operate remotely. Once all systems are operating independently, simultaneous operation shall be conducted with the aim to maintain controllability of the platform at all instances. Successful completion of this objective, once the platform is able to be operated remotely, is the first step towards full platform automation.  

\item Modification of an existing platform to accommodate a metal detector and a GPR. A sensor mount will be designed and built so that it can attach to the platform and support the chosen sensors safely, while adhering to the operating requirements of both the platform and sensors. The most suitable design will be chosen from several concepts based on the its ability to isolate the sensor suite and minimise sources of interference to the sensor output. On completion of this objective, the platform will be fully integrated with the sensor suite.

\item Development of software for platform automation and navigation, allowing for the platform to travel autonomously under supervision from an operator. Initially, automation and navigation software will be tested in a virtual environment, allowing for full simulation of operations in a risk-free manner. The outputs of this virtual platform will include the location, heading, and speed of the platform. On completion of objective 1,  the software and control systems will be integrated into the real platform, and further testing will be performed. After completing this objective, the platform will be fully autonomous.

\item Development of software for detection and autonomous classification of subsurface objects, with a focus on reducing the number of incorrect identifications or false positives. After initial testing, the output signals from the metal detector and GPR will be processed in real time. The results will be analysed with the aim of identifying suitable metrics that can be used to classify and differentiate various targets. A database of metrics will produced and used to identify the likelihood of a landmine being present. Operational trials will be conducted to test the effectiveness of the developed system. On completion of this objective, the sensor suite will be able to detect targets in real time.

\item Logging of detected objects with GPS coordinates. Once an object has been located, its position will be recorded and transmitted to a handheld device used by an operator. The output from the detection software in objective 4 will also be made available, allowing the operator to make an informed decision about how to proceed. Once the sensor suite has integrated with the platform (objective 2) and the platform is fully autonomous (objectives 1 and 3), this is the final objective required to complete the project goal.
\end{enumerate}

\subsection{Extended objectives}
\seclabel{extendedObjectives}
Extended objectives are those that are beyond the scope of the project, but may be attempted on completion of the primary objectives. These objectives are:

\begin{enumerate}
\item Automatic traversal of a designated area enclosed by a user defined region. Initially, single waypoint navigation will be conducted, requiring the platform to travel from one predefined point to another before coming to a stop. In this objective, waypoints enclosing an area will be used, requiring the platform to generate its own path to traverse the enclosed area. Software will be tested in a virtual environment before being implemented into the platform of completion of the primary objectives. Completion of this objective will extend the navigational capabilities of the platform, and make it easier for an operator to use. 

\item Provision of a live video feed as well as live updates of vehicle location and direction to the operator's handheld device. While the operator will be able to view the location of the platform using a handheld device, a live video feed allows for greater visualisation of operating conditions. This would also provide a greater ability to detect and respond appropriately to unknown obstacles in the search area.

\item Physical marking of landmine locations by the platform at the time of detection. GPS coordinates alone may not provide sufficient precision for identifying the location of landmines. An electromechanical system that physically marks the location would allow demining personnel to identify this location more clearly, and not be affected by potential calibration errors or systematic offsets from a GPS position.
\end{enumerate}

\section{Scenario of operation}
\seclabel{SoO}
This section will look at several real world environments that the platform could be expected to operate in, and presents the mission profile, which is the intended use case for the platform. Based on these, a set of performance specifications are developed.  

\subsection{Operating environments}
\seclabel{operatingEnvironments}
There are a large number of countries in which ADF personnel are deployed, each of which has differing terrain, vegetation, soil types, landmine types and sizes of landmine affected areas. Since it is impossible to develop a platform that suits all real world operating conditions, only two such environments are considered, both of which are heavily affected by landmines \parencite{AustralianGovernment2016}. 

\subsubsection{Eqypt}
\seclabel{Egypt}
Egypt is listed as the country most contaminated by landmines, with an area of roughly 25,000 square kilometres containing over 23 million buried anti-tank and anti-personnel mines \parencite{Rushfan2008}. The main contaminated areas are along the North coast and the Sinai Peninsula. These areas represent 22\% of the total area of Egypt and consist of barren shifting sands and lithosols with large flat expanses of sandy land \parencite{Nahrawy2011}. The chosen platform would then be required to travel over sandy terrains and over small obstacles such as rocks, with the sensors being able to detect landmines through sandy and loose soil. 
 \subsubsection{Iran}
 \seclabel{Iran}
Iran has an area of approximately 18,000 square kilometres of land contaminated with 12-16 million landmines left over from the 1980-1988 Iran-Iraq conflict \parencite{landmineMonitor2015}. The main areas of contamination are along the provinces of Khuzestan, which is mainly covered with sandy soils similar to those in Egypt. 

\textcolor{red}{This paragraph and the one before it need more information about terrain, vegetation, soil type, moisture, mine type, \textbf{mine depths}, obstacles, gradient, whatever else can be found! These should then lead provide the basis for the details presented at the start of the mission profile. Or maybe have an additional subsection, which describes the chosen operating environment... - RK}

\subsection{Mission profile}
\seclabel{MissionProfile}
A typical demining operation requires personnel to clear a path or area, ensuring it is safe for ground troops and vehicles to pass over. The mission profile developed for the platform resembles such an operation, and will form the basis of many of the project's design requirements.
\begin{quote}\textit{The mission takes place in a region with dry, sandy soils and level terrain unobstructed by vegetation or obstacles, such as a road or open field. A number of types of landmines, including high and low metal content anti-personnel and anti-tank mines, can be found buried at depths of up to 15 cm. The operator will use a handheld device to select waypoints that define a path or the boundaries of an area of interest. The platform will travel out to this path or area, then autonomously navigate it. Using the onboard sensor suite, the platform scans the area in real time at an operational speed of 5 km/h. Whenever a detection is made, the platform will stop, log the GPS coordinates of the location  and send sensor data and a confidence interval to the operator, giving the likelihood of the detected object being a landmine. Once the operator instructs the platform to resume, it continues to scan the area. Over time, a map of the scanned region showing all detected objects is developed using the sensor data. At any point the operator can abort the mission, or take manual control of the platform.}
\end{quote}

\subsection{Performance specifications}
\seclabel{performance}
Knowing the basic operating environment and mission profile, performance specifications can be defined, encompassing the platform and sensor systems.

\subsubsection{Platform}
\seclabel{platform}
In order to be able to autonomously navigate a region of interest and meet the objectives outlined earlier, the platform must meet the following requirements: 
\begin{itemize}
 \item Carry a total payload of 100 kg. A pair of metal detector and GPR arrays with their respective control boxes have an estimated gross weight of 80 kg. This allows 20 kg for any mounting structures,  electronics and computers. 
 \item Travel off-road in regions with dry sandy soils and level unobstructed terrain.
\item Travel in a straight line, both forwards and reverse, and perform turns while maintaining an operational speed of 5 km/h.
\item Be manoeuvrable enough to navigate the region of interest, yet controllable enough than in case of an emergency, it can come to a complete stop from its operational speed in 1 m.
\end{itemize}

\subsubsection{Sensors}
\seclabel{sensors} 
The requirements for the sensor system are as listed below:
\begin{itemize}
\item Detect high and low metal content anti-personnel and anti-tank mines.
\item Identify mines at depths of up to 15 cm at an operational speed of 5 km/h.
\item Discriminate between landmines and other objects, aiming to minimise the number of false positives detected.
\end{itemize}

\section{Project scope}
\seclabel{projectScope}

%Acceptance criteria: The conditions that must be met before project deliverables are accepted.
%Deliverables: The products, services, and/or results your project will produce (also referred to as objectives).
%Project Exclusions: Statements about what the project will not accomplish or produce.
%Constraints: Restrictions that limit what you can achieve, how and when you can achieve it, and how much achieving it can cost.
%Assumptions: Statements about how you will address uncertain information as you conceive, plan, and perform your project.

The aim of this project is to demonstrate the feasibility of a concept that incorporates an autonomous platform and a real time multisensor landmine detection system. It is not intended for the final design solution to be a commercially viable device that is ready to be deployed for the ADF. Rather, the challenges associated with the development of such a system, and the methods used to overcome these challenges, are perhaps the most valuable outcomes of the project. 

\textcolor{red}{The scope from the hovercraft report is only this big, but I guess we can add a few more relevant details about things that are outside of the scope (can't specifically think of anything right now) - RK}
\end{document}