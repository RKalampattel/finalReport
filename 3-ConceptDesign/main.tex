\documentclass[main.tex]{subfiles}
% \nomenclature[A]{GPR}{Ground Penetrating Radar}%

\begin{document}
\chapter{Concept Design}
\chaplabel{conceptDesign}

This chapter presents the design requirements and concept solutions for the various project subsystems, namely the sensor subsystem, including signal processing and sensor fusion, the platform subsystem, including automation and navigation, and the sensor mount subsystem.

\section{Sensor selection}
The selection of sensors for the project was based on the requirements from the scenario of operation. These are as follows:

\begin{itemize}
\item Detection of various target types: The sensor suite should be able to detect anti-personnel and anti-tank mines with high and low metal content 
\item Detection speed and depth: The sensors must be capable of real time detection at an operational speed of 5 km/h and depths of up to 15 cm
\item Accurate discrimination: The chosen sensors must be able to discriminate between landmines and other objects, with an aim to minimise false positives
\end{itemize}

As outlined in the benchmarking section, the primary sensors used for landmine detection operations are metal detectors and GPRs. When used in isolation, each sensor has several limitations. Metal detectors can only be used to detect metallic mines, they produce a large number of false positives, and their penetration depth is restricted. On the other hand, GPR systems produce data that is difficult to interpret, and is also very sensitive to environmental conditions such as soil type and moisture content. In both cases, signal processing of some form is required, either using software to identify detected objects, or with the involvement of a skilled operator.

Multisensor systems allow a greater range of targets to be identified. In addition, having two sensors provides the ability to confirm the presence of metallic mines, which reduces false positives and thereby increases the efficiency of operations. While it is difficult to quantify this increase in efficiency, this result has been demonstrated in experimental work carried out by \textcite{Takahashi10}. As can be seen in \Figref{FAR}, the false alarm rate, defined as the number of false alarms per unit area, for a system that combines a metal detector and GPR is roughly half that of a metal detector alone. Thus, a multisensor system meets the necessary requirements outlined above, while addressing some limitations of individual sensor systems.

\begin{figure}[ht]
\includegraphics[width=0.6\textwidth]{3-ConceptDesign/FAR.PNG}
\centering
\caption[Reduction in false alarm rate observed as a result of using a multisensor system]{Reduction in false alarm rate observed as a result of using a multisensor system (ALIS-MD+GPR) as opposed to a metal detector alone \parencite{Takahashi10}} \figlabel{FAR}
\end{figure}

Initially, it was expected that the sensor suite would be comprised of a metal detector array and a GPR array provided by the DSTG, with a total payload weight of 80 kg. The use of the 3 m wide arrays would allow for larger areas to be scanned in a shorter period of time, leading to an increase in operational efficiency. Due to unforeseen circumstances, the arrays could not be made available for the project, and so smaller line scan sensors had to be used instead. Off-the-shelf sensors were then considered, however most were not able to produce any raw data output that could be processed to meet the project objectives. In the case of GPR systems, the high cost of systems was also a limiting factor. Eventually, a pair of sensors were borrowed from the DSTG, the specifications of which are presented below. Despite the fact that line scan sensors were used, it was assumed that the scanning arrays would be made available by the completion of the project, and so the size and weight specifications of the arrays were used in selecting the platform and designing the sensor suite. 

\subsection{AMDS metal detector panel}
The AMDS is a metal detector panel produced by Minelab (\Figref{AMDS}). The device has three sets of transmitting coils and a single receiving coil, allowing for three channels of data to be recorded at once. Each channel operates at four frequencies, 1.5 kHz, 4.5 kHz, 13.5 kHz and 40.5 kHz, which means that in total up to 12 sets of data can be recorded simultaneously. This recorded data contains the real and imaginary parts of the voltage signal measured due to changes in magnetic flux in the target. The data is transferred to a computer through USB, where it can be seen in real-time using a LabVIEW virtual instrument called Dataspleen, or saved in CSV format. The metal detector requires a 10 volt power supply at 0.5 amps.  

\begin{figure}[ht]
\includegraphics[width=0.5\textwidth]{3-ConceptDesign/AMDS.JPG}
\centering
\caption{AMDS metal detector panel with power supply and control unit} \figlabel{AMDS}
\end{figure}

\subsection{SIRO-PULSE II GPR unit}
The SIRO-PULSE II is a handheld GPR unit developed by the CSIRO, originally intended for the inspection of buildings (\Figref{siro}). It is a pulsed radar that can operate in either bi-static or differential antenna configurations, producing three sets of data in total. The unit comes with three interchangeable antenna heads of frequencies 800 MHz, 1.4 GHz and 2 GHz, allowing the resolution and penetration characteristics of the radar to be varied. Data is transferred to a computer via USB, where it can be viewed in real-time with the accompanying SiroPuleII software program. The program produces outputs of the frequency waveform over time, which are collected together to produce a B-scan in real time. The unit is battery powered with a runtime of up to 10 hours. 

\begin{figure}[ht]
\includegraphics[width=0.5\textwidth]{3-ConceptDesign/SiroPulse.JPG}
\centering
\caption{SIRO-PULSE II GPR unit with control unit and antenna heads} \figlabel{siro}
\end{figure}

\section{Sensor fusion and signal processing}
Regardless of the fact that the sensors suite met the outlined requirements, there were several key challenges to be addressed, namely sensor fusion and signal processing. The first of these relates to how the data from the separate sensors can be consolidated to provide a single meaningful result, while the second involves processing the output from the sensors to allow an operator to identify targets. 

\subsection{Sensor fusion}
As discussed in the literature review, there are three primary levels of sensor fusion, decision level, feature level and data level \parencite{Yarovoy2009}. Of these, decision level sensor fusion was selected as it offers the simplest and least computationally intensive solution, since the outputs from the sensors can be processed individually. This method also lends itself well to machine learning techniques, allowing a target to be identified based on correlations with a known dataset. The main disadvantage of this method is that it is not as robust as the other methods, however this was not expected to be an issue given the scope of the project. 

To be able to implement decision level sensor fusion, a decision needs to be made with each sensor regarding whether or not the target is a mine. This decision, along with a confidence interval, can be passed onto the sensor fusion algorithm as shown in \Figref{fusion}, which assigns then makes the final decision based on a weighted evaluation of the data it has received. In order for each sensor to make a decision, metrics need to be identified which can be used to characterise targets. As a result, preliminary tests were carried with both sets of sensors to find such metrics.

\begin{figure}[ht]
\includegraphics[width=0.85\textwidth]{3-ConceptDesign/fusion.PNG}
\centering
\caption{Integration of signal processing and sensor fusion algorithms} \figlabel{fusion}
\end{figure}

\subsection{Metal detector output}
Initial tests with the metal detector were performed indoors. In the first set of tests, horizontal sweeps were made over a number of small metal samples (\Figref{samples}) placed on the floor, with the distance to the metal detector being kept constant at 32 cm. 

\begin{figure}[ht]
\includegraphics[width=0.8\textwidth]{3-ConceptDesign/samples.jpg}
\centering
\caption{Metal samples used for evaluation of metal detector outputs} \figlabel{samples}
\end{figure}

\Figref{metals} shows the results obtained from the second channel plotted in Matlab. The four lines in each plot correspond to the four frequency signals, while the axes represent the real and imaginary components of the voltage signal. It can be seen that for different metals, the magnitude and phase angle of the signal vary, as discussed in the literature review. 

\begin{figure}[ht]
\includegraphics[width=0.65\textwidth]{3-ConceptDesign/metals.PNG}
\centering
\caption{Phase plots for several different metal samples at 32 cm depth} \figlabel{metals}
\end{figure}

In the second set of tests, a single brass sample was scanned, with the distance to the metal detector changed between scans. The results are shown in \Figref{phaseDepth}, and while the phase angle does not vary with depth, the signal magnitude does. Based on on these tests, it was concluded that the magnitude and phase angle of the signal would be suitable metrics for characterising objects; the phase angle can be used to identify the metal, while the magnitude provides information about the depth, size and material. Thus, the aim of the signal processing algorithm for the metal detector is to find these metrics given the metal detector input data. 

\begin{figure}[ht]
\includegraphics[width=0.65\textwidth]{3-ConceptDesign/phaseDepth.PNG}
\centering
\caption{Phase plots for a brass sample scanned at different depths} \figlabel{phaseDepth}
\end{figure}

\subsection{GPR output}
Preliminary tests for the GPR could not be conducted at the same time as metal detector testing due to issues in operating the GPR. However, sample datasets were provided by the DSTG, and these were analysed in Matlab. \Figref{cans} shows the B-scans for an empty (background) scan, as well as a scan over two soft drink cans. 

Unlike with the phase plots obtained from the metal detector, it was not as clear what metrics could be identified from the GPR data. To further aid in visualising the signal, the background was subtracted (\Figref{cansNoBG}). This made it easier to identify features in the scan, the size and position of which could be found easily. Once the position of the feature is known, the A-scan corresponding to the feature can be extracted, and this can also be used to characterise the feature. Hence, the chosen metrics for the GPR are feature size, feature position and A-scan characteristics.

\begin{figure}[ht]
\centerline{
\begin{tabular}{cc}
\subfloat[]{\includegraphics[width=0.47\textwidth]{3-ConceptDesign/bg.PNG}}
& \subfloat[]{\includegraphics[width=0.47\textwidth]{3-ConceptDesign/cans.PNG}}\\
\end{tabular}}
\caption{Background scan (a) and B-scan over two soft drink cans, circled in red (b)} 
\figlabel{cans}
\end{figure}

\begin{figure}[!ht]
\includegraphics[width=0.6\textwidth]{3-ConceptDesign/cansNoBG.PNG}
\centering
\caption{B-scan for soft drink cans with background removed} \figlabel{cansNoBG}
\end{figure}

\section{Platform selection}
\seclabel{platformselect}
The platform is used to mount the required detection equipment and acts as the foundation for a completely autonomous system. The design and construction of a new platform was considered to be outside the scope of the project, and so the modification of an existing platform was the preferred approach. The requirements for the platform primarily come from the scenario of operation, but also include some other considerations: 

\begin{itemize}
 \item Payload: The platform should be able to carry a total payload of 100 kg, and must allow for this payload to be mounted.
 \item Terrain traversing: The platform should be able to travel off-road in regions with dry sandy soils and level unobstructed terrain.
\item  Operational speed and acceleration: The platform should be able travel in a straight line, both forwards and reverse, and perform turns while maintaining an operational speed of 5 km/h.
\item Manoeuvrability and controllability: The platform should be manoeuvrable enough to navigate a region of interest, with a smaller turning angle preferred, yet controllable enough that in case of an emergency, it can come to a complete stop from its operational speed in 1 m.
\item Cost and availability: The platform should have a reasonable cost given the project budget of \$16,500, ans should be available for use as early as possible. 
\item Ease of automation: The platform should be sufficiently simple to automate, with a preference given to platforms that have the framework for automation already implemented. 
\item Ease of transportation: The platform should be sufficiently easy to transport from a workshop to a testing locations using a ute or trailer.
\end{itemize}

The detailed platform selection process is outlined in \Chapref{platformSelectionApp}. Several platforms were considered and evaluated against the above requirements, and the most appropriate platform was found to be an autonomous quad bike provided by the DSTG (\Figref{quadbike}). The quad bike is a Honda TRX450r, and was previously modified at the University of Adelaide on several occasions, most recently by \textcite{scheiner2011}. 

\begin{figure}[ht]
\includegraphics[width=0.6\textwidth]{3-ConceptDesign/bike.JPG}
\centering
\caption{DSTG autonomous quad bike} \figlabel{quadbike}
\end{figure}

The intended application for the quad bike was to autonomously drive through vineyards. In order to achieve this, a number of systems had to be modified, namely the steering, throttle, brake and gear change systems \parencite{scheiner2011}. The steering is controlled by an Animatics Smart Motor, connected to the steering column using a v-belt. A servo motor is to control the throttle, while a wheel speed encoder measures the vehicle's position. The linear actuator is used to control the brakes, with a strain gauge used to measure the braking intensity. Finally, two linear potentiometers are used to measure the position of a linear actuator in the gear system. A Dragon Board is used as the main controller, handling communications between all subsystems. 

\section{Automation and navigation}
The navigation and automation systems are responsible for instructing the quad bike as to what path it should follow, which is achieved through the control of the various actuators. As stated in the scenario of operation, platform navigation will be primarily handled using waypoints. At the start of a mission, the operator will select waypoints that define either a path, or the boundaries of some region of interest. If a region is selected, it is broken down into a series of waypoints which, once connected, will form a path for the quad bike. In the alternate use case, the navigation system will operate based directly on waypoints created from a user defined path.

\subsection{Path tracking}
\seclabel{pathconcept}
As the platform will be primarily performing two tasks, following a low curvature curve (straight line) and turning a specified angle, the navigation system will define the path using a 'piecewise linear path' discussed in \secref{pathTrackingLitReview}. \Textcite{snider2009} provides an empirical comparison of path following algorithms, shown in \Figref{trackingComparison}. Tracking methods are ranked by implementation difficulty, from least difficult to most difficult. \Textcite{snider2009} goes on to describe recommended applications for each method. Pure Pursuit is ideal in situations for slow driving and/or on discontinuous paths, the Stanley method for smooth highway driving and/or parking manoeuvres, the Kinematic model for smooth parking manoeuvres, and the Dynamic model for highway driving at speed. Based on the scenario of operation, which involves slow driving on discontinuous paths, the Pure Pursuit method is the most appropriate option, and will be used as the tracking method.
\begin{figure}[ht]
\includegraphics[width = \textwidth]{3-ConceptDesign/pathTrackingSummary2.png}
\centering
\caption[Empirical comparison of path tracking algorithms]{Empirical comparison of path tracking algorithms \parencite{snider2009}} \figlabel{trackingComparison}
\end{figure}

\subsection{Performing turning manoeuvres}
During scanning of an area, whenever the platform reaches a boundary, it will be required to turn some specified angle within a small area defined by the width of the 3 m sensor arrays. This prevents the platform from travelling over unscanned areas and potentially detonating a landmine. Alternatively, if the scan history for the surrounding region is available to the platform in the form of a map, a larger area may be available to conduct the turn. Due to there being little or no literature available on this unique topic, an adaptation of the simplified Ackermann model will be used in conjunction with the Stanley model, which excels for parking manoeuvres \parencite{snider2009}, to achieve the desired turn.

\subsection{Positioning}
The positioning system is used to determine the location of the quad bike for navigation, and to send that data to the operator's handheld device when a landmine is detected. Real time positioning in remote locations and accuracy to within 0.5 m is required from the system to meet the needs of the scenario of operation. A number of options were looked at when designing the positioning system including Local Positioning Systems (LPS), Global Positioning Systems (GPS), and dead reckoning techniques.
\nomenclature[A]{GPS}{Global Positioning System}%
\nomenclature[A]{LPS}{Local Positioning System}%

LPS use three or more signalling beacons of known location to determine a position through triangulation. These systems can be highly accurate, however they require the signalling beacons to be stationed near the area of interest. Transporting and setting up at least three signalling beacons is a time consuming process and is particularly inefficient when numerous areas of interest are required to be scanned. Since personnel are needed to place the beacons in potentially dangerous area, this method was not considered any further. 

Similar to LPS, GPS measurements are very accurate. However, due to the distance between receivers and a number of linked effects, the positional data is very noisy, providing location reliably within only 4 metres. Better accuracy can be achieved by using correction techniques, such as Real Time Kinematics (RTK), which can provide sub-centimetre accuracy. Similar to LPS, correction methods like RTK require base locations within 15 kilometres of the GPS and can take many seconds to receive a fix, both of which make it difficult to achieve real time positioning in remote areas.
\nomenclature[A]{RTK}{Real Time Kinematics}%

Dead reckoning techniques calculate the current position by advancing a previous position, based on a known speed and heading over some time step. These techniques are highly prone to cumulative error known as drift. A widely used application of dead reckoning is in Inertial Measurement Units (IMU), where accelerometers and gyroscopes are used to determine linear and rotational accelerations and thus have information to update position. This method is able to provide positional data in real time, however over longer time periods the accuracy becomes unreliable.
\nomenclature[A]{IMU}{Inertial Measurement Unit}%

To correct for drift from dead reckoning techniques, GPS aided Inertial Navigation Systems (INS) can be used. Accurate but noisy data from a GPS is constantly fed into an estimation algorithm alongside a smooth, but error accumulating position to correct for the drift. Kalman Filtering is one such technique which is used to combine information from various sensors to provide a much more reliable estimate of position.
\nomenclature[A]{INS}{Inertial Navigation System}%

\section{Sensor mount}
\seclabel{sensormount}
The sensor mount is required to support the weight of both the GPR and the metal detector, while being sturdy enough to isolate the sensors from vibrations caused by the platform. The following section describes the requirements imposed on the mount, the selection of appropriate materials and initial designs.  

\subsection {Sensor requirements} 
The design of the mount must satisfy two operating requirements for each sensor, the maximum operating height above the ground and the minimum proximity to other objects to prevent interference. 

Sensitivity tests with the AMDS metal detector panel found that metal objects at distances greater than 35 cm could not be detected, regardless of material or size. Based on this, a minimum proximity of 40 cm was specified. To maximise sensing depth, the metal detector should be placed as close to the ground as possible. However, resting the detector on the ground may result in damage if the platform vibrates excessively, or a small obstacle is encountered. Taking this into account, a maximum ground clearance of 5 cm was specified.

Tests were also carried out with the SIRO-PULSE II GPR unit, resulting in negligible interference from nearby all objects. Thus, a minimum proximity was not specified for the GPR. In order to reduce the effect of the air-ground interface and other sources of interference, the GPR should operate as close to the ground as possible. If the wheel encoder on the GPR is to be used while in operation, the GPR must be in physical contact with the ground for the duration of scanning. Taking these factors into account, a maximum ground clearance of 1 cm was specified. 

\subsection {Material selection}  
The main requirement for the mount was that a non-metallic material was used, to meet the constraints of the metal detector. Other considerations included good stiffness and vibrational characteristics, minimal mass, a low cost and good availability and workability. After a detailed material selection process, described in \Chapref{sensorMaterialsApp}, wood, bamboo, carbon fibre reinforced polymers (CFRP) and glass fibre reinforced polymers (GFRP) were found to be the most appropriate materials. 

While bamboo is the best material to use for the sensor mount frame based on it's vibrational characteristics and stiffness alone, it is difficult to obtain in Australia in structural form. Wood is the next best option, being very easy to work with and readily available. GFRP has a similar stiffness to wood but with worse damping properties and a higher density. On the other hand, CFRP has a much better stiffness however it sacrifices damping properties and density. The two composites are also more difficult to work with, and are more expensive. Hence, the best material to use for the sensor mount is wood. In order to meet the stiffness requirements, a structural 
 
\nomenclature[A]{CFRP}{Carbon fibre reinforced polymer}%
\nomenclature[A]{GFRP}{Glass fibre reinforced polymer}%

\subsection{Steering focal point}
Both the GPR and metal detector are required to be mounted to the platform in a cantilever-like fashion in order to avoid contact with the ground where it would risk detonation of a landmine. Each of the sensor arrays is expected to weigh between 10 kg and 30 kg.
Mounting needs to be done in a configuration that will keep individual array sensors aligned so that corresponding ground information can be collated and analysed correctly. If the assumption of Ackermann steering holds for the quad bike we can use the simplified, 2-wheel model to assess the location of the sensors relative to the quad bike. The simplified model provides the following geometric relationship:
$$
tan(\delta) = \frac{L}{R}
$$
Where $\delta$ is the steering angle, $L$ is the wheelbase and $R$ is the radius of the arc that the centre of the rear axle will follow. This simplification has been used to perform simulations on the steering effectiveness of the vehicle and will also be used to estimate turning capabilities as part of the route planning software.
Initial simulations showed that front mounted sensors with rear-wheel steering is the best option for the application with far superior coverage results compared to front-wheel steering. Further results are shown in \Figref{turnCoverage}
\begin{itemize}
\item Trial 1, represented as the leftmost diagram in \Figref{turnCoverage}, shows the turning example for a case where the turning wheels are 1.5 metres in front of the focal point of the turn, or front wheel steering. The sensor array is 0.5 metres in front of the turning wheels, 2 metres from the focal point of the turn. Evident from the figure is how the coverage area does not form a tight loop and the coverage area does not include the path of the vehicle. This is critical as it will mean the quad bike will be passing over ground not scanned by the sensors.
\item Trial 2, represented by the middle diagram in \Figref{turnCoverage}, shows the results for a case where the turning wheels are 1.5 metres behind the focal point of the turn, or rear wheel steering. The sensor array is placed 0.5 metres in front of the focal axle. This trial shows a much tighter scanning loop than the previous test and the path of the quad bike is entirely within the scanned area. This is ideal.
\item The third trial, represented by the rightmost diagram in \Figref{turnCoverage}, is purely for comparison and shows the coverage area of a vehicle with turning wheels 1.5 metres in front of the focal axle, but with the sensing array only 0.5 metres in front of the focal axle. This would place the sensing array between the two axles of the vehicle. Under this arrangement an identical scanning path is achieved using a front-wheel steering setup. This is not useful for the application as the sensors must be in front of the vehicle to detect landmines.
\end{itemize}
% Split, subfig
\begin{figure}[ht]
\includegraphics[width=0.8\textwidth]{3-ConceptDesign/Detector_Coverage.png}
\centering
\caption{Landmine detector coverage simulations} \figlabel{turnCoverage}
\end{figure}
This result is expected. Referring to \Figref{geometricBicycleModel}, the turning radius of the rear wheel is $R$ and the turning radius of the front wheel is the hypotenuse of the triangle formed – of length $\sqrt{R^2 + L^2}$, which is greater than $R$. If the sensing array was some distance $q$ in front of the front wheels, its turning radius would be $\sqrt{R^2 + [L+q]^2}$, greater yet again. However, if the sensing array was an equal distance $q$ behind the rear wheels (or front wheels if this was a rear-steering arrangement) then the turning radius of the sensors would be $\sqrt{R^2 + q^2}$. From this it can be seen that to minimise the effective turning radius then $q$ should be minimised.

The sensor arrays will be mounted at the 'rear' of the quad bike which will be primarily driven in reverse to achieve the desired scan coverage. The mounting bracket for the arrays will be constructed on-the-fly with the assistance of the workshop and its technicians. The bracket should be constructed from PVC and meet the following specifications:
\begin{itemize}
\item Metal detector clear of metal in a 500 mm radius and 2000 mm vertically.
\item GPR clear of obstructions in a 100 mm radius and 50 mm vertically.
\end{itemize}

\subsection{Initial designs}
Basic design for the sensor mount can be realised with the sensor requirements. Using the DSTG Quad bike as the chosen platform, preliminary designs can be sketched. Considerations for both front and rear mounted sensor arrays were considered. Design 1, 2 and 3 trial different mounting techniques and frame designs.



\subsubsection{Design 1}
Design one is represented in \Figref{design1} and is a rear mounted frame. This mount assumes that the quad bike will be more manoeuvrable when steered from the rear, requiring the direction of travel to be in reverse. Thus, the frame mounting on the back is necessary. 
 \begin{figure}[ht]
 \includegraphics[width=.5\textwidth]{3-ConceptDesign/Rear_Mount.png}
 \centering
 \caption{Rear frame mount concept design}
 \figlabel{design1}
 \end{figure}

\subsubsection{Design 2}
Design 2 represents a front mounted frame assuming that the quad bike is driven in a standard configuration as shown in \Figref{design2}. The GPR is placed in close proximity to the body with the metal detector attached further in front. This design appears to be the smallest and uses the least amount of material.
\begin{figure}[ht]
\includegraphics[width=0.5\textwidth]{3-ConceptDesign/front_frame_design_triangle.png}
\centering
\caption{Front frame mount concept design}
\figlabel{design2}
\end{figure}

\subsubsection{Design 3}
Design 3 incorporates a staggered design with the metal detector and GPR placed further in front of the platform as shown in \Figref{design3}. This larger stand off distance would allow for less violent stops as the allowable stopping distance would be increased. However, the required material and weight would be greater potentially requiring greater reinforcements to mitigate flexing and warping. These reinforcements could be from the use of fibreglass coating over the PVC piping.
\begin{figure}[ht]
\includegraphics[width=0.5\textwidth]{3-ConceptDesign/front_mount_staggered.png}
\centering
\caption{Front frame mount concept design}
\figlabel{design3}
\end{figure}\\

The sensor mount will be made from a combination of PVC and fibreglass. The fibreglass will be supporting sections to prevent flexing and warping. As exact dimensions and platform performance characteristics are unknown, specific mounting orientation and size cannot be decided upon until the platform is delivered. Once the Platform has been delivered for the project, specific and more detailed designs can be created

\section{Subsystem integration}

\seclabel{conceptprojectdesign}

\subsection{Electronics}
Requirements for the electronics subsystems can be inferred from the project aims. As a major component of the project would be attempting advanced signal processing methods in real time, significant processing capabilities will be required on the mobile platform. In addition to this, the signal processing must be capable of running simultaneously with a number of other platform software systems, such as vehicle control and telemetry. To achieve this without requiring multiple discrete hardware components (which would require communications input/output (I/O) interfaces to share data), a single hardware system capable of executing multiple threads simultaneously and asynchronously is required on the vehicle. The hardware system executing the signal processing software must also be capable of reading sensory input from USB devices, as this is the communications format available on the GPR units provided by the DTSG. 
\nomenclature[A]{I/O}{Input/Output}% 

A second major component of the project is the automation of the remote vehicle, requiring software control over a series of actuators and sensors. To provide the greatest fidelity of control over the vehicle the electronics hardware used to interface with the actuators and sensors must be capable of reading and writing to low-level I/O devices quickly and with minimal latency or overhead. To achieve this, the ability to create hardware interrupts are desirable, which would allow the software to process incoming sensory data as soon as it is received. The software for the parsing and decoding of raw input signals to generate usable information is not expected to be complicated, as so the hardware will not need to be particularly advanced or have capacity for high speed processing.

\begin{itemize}
\item \textbf{Bespoke Electronics}\\
Bespoke electronic equipment has the capacity to allow incredibly fast access to I/O devices through the use of task-specific commercial off-the-shelf (COTS) chips. However, the time consumption and expense of planning an entirely hardware-driven control system for anything more than trivial data handling is inappropriate for this project. The inability to prototype as with software means that the ability to test and then revisit a solution is not possible, and a hardware/purely electronics driven system is not capable of general purpose processing. Therefore, this is not a realistic option for achieving the project aims.
\nomenclature[A]{COTS}{Commercial off-the-shelf}% 
\item \textbf{Microcontrollers}\\
Microcontrollers have become the de facto standard for small to medium software-based projects which require access to physical sensors and actuators, due to their readily available access to low level I/O. Microcontrollers supporting common languages such as C++ and Java allow easy development and rapid prototyping, though the inability to easily connect debugging equipment or generate test output slows the development process. Microcontrollers are inexpensive and provide high I/O availability but at the cost of limited processing power. The low-level nature of microcontrollers means that desirable features like hardware interrupts are exposed to and accessible by developers. 
\item \textbf{Desktop computing equipment/Laptop} \\
Conventional desktop computing equipment is the fastest general purpose computing hardware that will be available to the project. In addition to having the greatest computing power, it has the highest ability to support prototyping and allows for rapid software development with readily accessible software generation and debugging tools. The drawback of this higher-level computing platform is the reduced accessibility of low-level I/O devices, and the amount of computing overhead caused by operating system processes. Operations that require fast I/O access may be hampered by the inability to ensure thread availability, and so for robust operation this may require buffering to a secondary, lower level device.
\end{itemize}

None of the individual items presented allow for the full range of requirements of this project. As a result, the general concept for the hardware arrangement to execute the software systems is shown below in \Figref{hardwareLayout}.
% is this figure text fucking big enough maziar??
\begin{figure}[ht]
\includegraphics[width = \textwidth]{3-ConceptDesign/electronics.png}
\centering
\caption{Conceptual hardware layout} \figlabel{hardwareLayout}
\end{figure}

Under this system, the project will use standard desktop computing equipment for the bulk of the software, to make use of its superior processing power and the rapid development it allows. This device will be the data handler and processor, and act as the 'central' software location for the project. Sensors and actuators that require low level I/O access will be connected to a secondary microcontroller, which will act independently to buffer inputs and outputs of the system, which can then be communicated to the primary computer over a serial communications connection. The project will not aim to develop any custom electronics boards and handle all signal amplification or processing in software.

\begin{figure}[ht]
\includegraphics[width=\textwidth]{3-ConceptDesign/fyp_structure.png}
\centering
\caption{Integration of automation, navigation, sensors and signal processing} 
\figlabel{central}
\end{figure}

\subsection{Software architecture}
The autonomous quad bike will be used as the platform on which the sensor mount and software subsytems will operate. A concept is shown in \Figref{frontConcept} and \Figref{rearConcept}. To achieve the framework for overall automation of the platform the integration of the actuator electronics, navigation, sensors, signal processing and operator device is required. This framework is completed through the Central Hub as shown in \Figref{central} where dotted lines indicate implementations of the interfaces. The Central Hub will combine and read all the relevant processes and transmit the required information to the operator device. This enables all processes to be completed in parallel, satisfying the deliverables as defined in \secref{primary}.

\begin{figure}[ht]
\includegraphics[width = 0.8\textwidth]{3-ConceptDesign/ConceptFront.jpeg}
\centering
\caption{Front wheel drive platform concept design} \figlabel{frontConcept}
\end{figure}
\begin{figure}[ht]
\includegraphics[width = 0.8\textwidth]{3-ConceptDesign/ConceptRear.jpeg}
\centering
\caption{Rear wheel drive platform concept design} \figlabel{rearConcept}
\end{figure}

The DSTG quad bike provides a basis to apply the completed concept designs and build upon them in order to have detailed designs for the automation and signal processing. 

\subsection{Communications}

\end{document}