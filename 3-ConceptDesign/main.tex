\documentclass[main.tex]{subfiles}
% \nomenclature[A]{GPR}{Ground Penetrating Radar}%

\begin{document}
\chapter{Conceptual Design}
\chaplabel{conceptDesign}
This chapter presents the design requirements and concept solutions for the various project subsystems, namely the sensor subsystem, including signal processing and sensor fusion, the platform subsystem, including automation and navigation, and the sensor mount subsystem. 

\section{Sensor system}
The requirements for the sensor system, as derived from the scenario of operation, are:
\begin{itemize}
\item Detection of various target types: The sensor suite should be able to detect anti-personnel and anti-tank mines with high and low metal content 
\item Detection speed and depth: The sensors must be capable of real time detection at an operational speed of 5 km/h and depths of up to 15 cm
\item Accurate discrimination: The chosen sensors must be able to discriminate between landmines and other objects, with the aim to minimise false positives
\end{itemize}
It was initially expected that the sensor suite would be comprised of a metal detector array and a GPR array provided by the DSTG, with a total payload weight of 80 kg. The use of the 3 m wide arrays would allow for larger areas to be scanned in a shorter period of time, leading to an increase in operational efficiency. Due to unforeseen circumstances, the arrays could not be made available for the project, and so smaller line scan sensors had to be used instead. Off-the-shelf sensors were considered, however most were not able to produce any raw data output that could be processed to meet the project objectives. In the case of GPR systems, the high cost of systems was also a limiting factor. Eventually, a pair of sensors were borrowed from the DSTG, the AMDS metal detector panel and the SIRO-PULSE II GPR unit. The specifications of these sensor are presented in \Chapref{sensorApp}. Further testing would be required to ensure the the sensors met the requirements outlined above. 

Despite the fact that line scan sensors were used, it was assumed that the scanning arrays would be made available by the completion of the project, and so the size and weight specifications of the arrays were used in selecting the platform and designing the sensor mount. 

\section{Sensor fusion and signal processing}
\seclabel{signalConcept}
Two of the key challenges to be addressed in analysing the output from the sensors were sensor fusion and signal processing. The first of these relates to how the data from the separate sensors can be consolidated to provide a single meaningful result, while the second involves processing the output from the sensors to allow an operator to identify targets. 

\subsection{Sensor fusion}
As discussed in the literature review, there are three primary levels of sensor fusion, decision level, feature level and data level \parencite{Yarovoy2009}. Of these, decision level sensor fusion was selected for the project as it offers the simplest and least computationally intensive solution, since the outputs from the sensors can be processed individually. This method also lends itself well to machine learning techniques, allowing a target to be identified based on correlations with a known dataset. The main disadvantage of this method is that it is not as robust as the other methods, however this was not expected to be an issue given the scope of the project. 

To be able to implement decision level sensor fusion, a decision needs to be made with each sensor regarding whether or not the target is a landmine. This decision, along with a confidence interval, can be passed onto the sensor fusion algorithm as shown in \Figref{fusion}, which assigns then makes the final decision based on a weighted evaluation of the data it has received. In order for each sensor to make a decision, metrics need to be identified which can be used to characterise targets. As a result, preliminary tests were carried with both sets of sensors to find such metrics.

\begin{figure}[ht]
\includegraphics[width=\textwidth]{3-ConceptDesign/fusion.PNG}
\centering
\caption{Integration of signal processing and sensor fusion algorithms} \figlabel{fusion}
\end{figure}

\subsection{Metal detector output}
Initial tests with the metal detector were performed indoors. In the first set of tests, horizontal sweeps were made over a number of small metal samples (\Figref{samples}) placed on the floor, with the distance to the metal detector being kept constant at 32 cm. 

\begin{figure}[ht]
\includegraphics[width=0.8\textwidth]{3-ConceptDesign/samples.jpg}
\centering
\caption{Metal samples used for evaluation of metal detector outputs} \figlabel{samples}
\end{figure}

\Figref{metals} shows the results obtained from the second channel plotted in Matlab, corresponding to the received signal at the centre of the metal detector panel. The four lines in each plot correspond to the four frequency signals, while the axes represent the real and imaginary components of the voltage signal. It can be seen that for different metals, the magnitude and phase angle of the signal vary, as discussed in the literature review. 

\begin{figure}[ht]
\includegraphics[width=\textwidth]{3-ConceptDesign/metals.PNG}
\centering
\caption{Phase plots for several different metal samples at 32 cm depth} \figlabel{metals}
\end{figure}

In the second set of tests, a single brass sample was scanned, with the distance to the metal detector changed between scans. The results are shown in \Figref{phaseDepths}, and while the phase angle does not vary with depth, the signal magnitude does. Based on on these tests,   it was concluded that the magnitude and phase angle of the signal would be suitable metrics for characterising objects; the phase angle can be used to identify the metal, while the magnitude provides information about the depth, size and material. Thus, the aim of the signal processing algorithm for the metal detector is to find these metrics given the metal detector input data. 

\begin{figure}[ht]
\includegraphics[width=\textwidth]{3-ConceptDesign/phaseDepth.PNG}
\centering
\caption{Phase plots for a brass sample scanned at different depths} \figlabel{phaseDepths}
\end{figure}

\subsection{GPR output}
Preliminary analysis of the GPR output could not be conducted at the same time as metal detector testing due to issues in operating the GPR. However, sample datasets were provided by the DSTG, and these were analysed in Matlab. \Figref{cans} shows the B-scans for an background scan, as well as a scan over two soft drink cans. 

Unlike with the phase plots obtained from the metal detector, it was not as clear what metrics could be identified from the GPR data. To further aid in visualising the signal, the background was subtracted (\Figref{cansNoBG}). This made it easier to identify features in the scan, the size and position of which could be found easily. As a result, it was decided that metrics used for the GPR would be feature size and feature position.

\begin{figure}[ht]
\centerline{
\begin{tabular}{cc}
\subfloat[]{\includegraphics[width=0.47\textwidth]{3-ConceptDesign/bg.PNG}}
& \subfloat[]{\includegraphics[width=0.47\textwidth]{3-ConceptDesign/cans.PNG}}\\
\end{tabular}}
\caption[Background scan over two soft drink cans]{Background scan (a) and B-scan over two soft drink cans, circled in red (b)} 
\figlabel{cans}
\end{figure}

\begin{figure}[!ht]
\includegraphics[width=0.6\textwidth]{3-ConceptDesign/cansNoBG.PNG}
\centering
\caption{B-scan for soft drink cans with background removed} \figlabel{cansNoBG}
\end{figure}

\section{Platform selection}
\seclabel{platformselect}
The platform is used to mount the required detection equipment and acts as the foundation for a completely autonomous system. The design and construction of a new platform was considered to be outside the scope of the project, and so the modification of an existing platform was the preferred approach. The requirements for the platform primarily come from the scenario of operation, but also include some other considerations: 

\begin{itemize}
 \item Payload: The platform should be able to carry a total payload of 100 kg, and must allow for this payload to be mounted.
 \item Terrain traversing: The platform should be able to travel off-road in regions with dry sandy soils and level unobstructed terrain.
\item  Operational speed and acceleration: The platform should be able travel in a straight line, both forwards and reverse, and perform turns while maintaining an operational speed of 5 km/h.
\item Manoeuvrability and controllability: The platform should be manoeuvrable enough to navigate a region of interest, with a smaller turning angle preferred, yet controllable enough that in case of an emergency, it can come to a complete stop from its operational speed before detonating a landmine.
\item Cost and availability: The platform should have a reasonable cost given the project budget of \$16,500, and should be available for use as early as possible. 
\item Ease of automation: The platform should be sufficiently simple to automate, with a preference given to platforms that have the framework for automation already implemented. 
\item Ease of transportation: The platform should be sufficiently easy to transport from a workshop to a testing locations using a ute or trailer.
\end{itemize}

The detailed platform selection process is outlined in \Chapref{platformSelectionApp}. Several platforms were considered and evaluated against the above requirements, and the most appropriate platform was found to be an autonomous quad bike provided by the DSTG (\Figref{quadbike}). The quad bike is a Honda TRX450r, and was previously modified at the University of Adelaide on several occasions, most recently by \textcite{scheiner2011}. 

\begin{figure}[ht]
\includegraphics[width=0.6\textwidth]{3-ConceptDesign/bike.JPG}
\centering
\caption{DSTG autonomous quad bike} \figlabel{quadbike}
\end{figure}

The intended application for the quad bike was to autonomously drive through vineyards. In order to achieve this, a number of systems had to be modified, namely the steering, throttle, brake and gear change systems \parencite{scheiner2011}. The steering is controlled by an Animatics Smart Motor, connected to the steering column using a v-belt. A servo motor is to control the throttle, while a wheel speed encoder measures the vehicle's position. The linear actuator is used to control the brakes, with a strain gauge used to measure the braking intensity. Finally, two linear potentiometers are used to measure the position of a linear actuator in the gear system. A Dragon Board is used as the main controller, handling communications between all subsystems. 

\section{Automation and navigation}
The navigation and automation systems are responsible for instructing the quad bike as to what path it should follow, which is achieved through the control of the various actuators. As stated in the scenario of operation, platform navigation will be primarily handled using waypoints. At the start of a mission, the operator will select waypoints that define either a path, or the boundaries of some region of interest. If a region is selected, it is broken down into a series of waypoints which, once connected, will form a path for the quad bike. In the alternate use case, the navigation system will operate based directly on waypoints created from a user defined path.

\subsection{Path tracking}
\seclabel{pathconcept}
As the platform will be primarily performing two tasks, following a low curvature curve (straight line) and turning a specified angle, the navigation system will define the path using a 'piecewise linear path' discussed in \secref{pathTrackingLitReview}. \Textcite{snider2009} provides an empirical comparison of path following algorithms, shown in \Figref{trackingComparison}. Tracking methods are ranked by implementation difficulty, from least difficult to most difficult. \Textcite{snider2009} goes on to describe recommended applications for each method. Pure Pursuit is ideal in situations for slow driving and/or on discontinuous paths, the Stanley method for smooth highway driving and/or parking manoeuvres, the Kinematic model for smooth parking manoeuvres, and the Dynamic model for highway driving at speed. Based on the scenario of operation, which involves slow driving on discontinuous paths, the Pure Pursuit method is the most appropriate option, and will be used as the tracking method.
\begin{figure}[ht]
\includegraphics[width = \textwidth]{3-ConceptDesign/pathTrackingSummary2.png}
\centering
\caption[Empirical comparison of path tracking algorithms]{Empirical comparison of path tracking algorithms \parencite{snider2009}} \figlabel{trackingComparison}
\end{figure}

\subsection{Performing turning manoeuvres}
During scanning of an area, whenever the platform reaches a boundary, it will be required to turn some specified angle within a small area defined by the width of the 3 m sensor arrays. This prevents the platform from travelling over unscanned areas and potentially detonating a landmine. Alternatively, if the scan history for the surrounding region is available to the platform in the form of a map, a larger area may be available to conduct the turn. Due to there being little or no literature available on this unique topic, an adaptation of the simplified Ackermann model will be used in conjunction with the Stanley model, which excels for parking manoeuvres \parencite{snider2009}, to achieve the desired turn.

\subsection{Positioning}
The positioning system is used to determine the location of the quad bike for navigation, and to send that data to the operator's handheld device when a landmine is detected. Real time positioning in remote locations and accuracy to within 0.5 m is required from the system to meet the needs of the scenario of operation. A number of options were looked at when designing the positioning system including Local Positioning Systems (LPS), Global Positioning Systems (GPS), and dead reckoning techniques.
\nomenclature{GPS}{Global Positioning System}%
\nomenclature{LPS}{Local Positioning System}%

LPS use three or more signalling beacons of known location to determine a position through triangulation. These systems can be highly accurate, however they require the signalling beacons to be stationed near the area of interest. Transporting and setting up at least three signalling beacons is a time consuming process and is particularly inefficient when numerous areas of interest are required to be scanned. Since personnel are needed to place the beacons in potentially dangerous area, this method was not considered any further. 

Similar to LPS, GPS measurements are very accurate. However, due to the distance between receivers and a number of linked effects, the positional data is very noisy, providing location reliably within only 4 metres. Better accuracy can be achieved by using correction techniques, such as Real Time Kinematics (RTK), which can provide sub-centimetre accuracy. Similar to LPS, correction methods like RTK require base locations within 15 kilometres of the GPS and can take many seconds to receive a fix, both of which make it difficult to achieve real time positioning in remote areas.
\nomenclature{RTK}{Real Time Kinematics}%

Dead reckoning techniques calculate the current position by advancing a previous position, based on a known speed and heading over some time step. These techniques are highly prone to cumulative error known as drift. A widely used application of dead reckoning is in Inertial Measurement Units (IMU), where accelerometers and gyroscopes are used to determine linear and rotational accelerations and thus have information to update position. This method is able to provide positional data in real time, however over longer time periods the accuracy becomes unreliable.
\nomenclature{IMU}{Inertial Measurement Unit}%

To correct for drift from dead reckoning techniques, GPS aided Inertial Navigation Systems (INS) can be used. Accurate but noisy data from a GPS is constantly fed into an estimation algorithm alongside a smooth, but error accumulating position to correct for the drift. Kalman Filtering is one such technique which is used to combine information from various sensors to provide a much more reliable estimate of position.
\nomenclature{INS}{Inertial Navigation System}%

\section{Sensor mount}
\seclabel{sensormount}
The sensor mount is required to support the weight of both the GPR and the metal detector, while being sturdy enough to isolate the sensors from vibrations caused by the platform. The following section describes the requirements imposed on the mount, the selection of appropriate materials and mounting location on the quad bike, as well as initial designs.  

\subsection {Sensor requirements} 
The design of the mount must satisfy two operating requirements for each sensor, the maximum operating height above the ground and the minimum proximity to other objects to prevent interference. 

Sensitivity tests with the AMDS metal detector panel found that metal objects at distances greater than 35 cm could not be detected, regardless of material or size. Based on this, a minimum proximity of 40 cm was specified. To maximise sensing depth, the metal detector should be placed as close to the ground as possible. However, resting the detector on the ground may result in damage if the platform vibrates excessively, or a small obstacle is encountered. Taking this into account, a maximum ground clearance of 5 cm was specified.

Tests were also carried out with the SIRO-PULSE II GPR unit, resulting in negligible interference from nearby all objects. Thus, a minimum proximity was not specified for the GPR. In order to reduce the effect of the air-ground interface and other sources of interference, the GPR should operate as close to the ground as possible. If the wheel encoder on the GPR is to be used while in operation, the GPR must be in physical contact with the ground for the duration of scanning. Taking these factors into account, a maximum ground clearance of 1 cm was specified. 

\subsection {Material selection}  
The main requirement for the mount was that a non-metallic material was used to meet the constraints of the metal detector. Other considerations included good stiffness and vibrational characteristics, minimal mass, a low cost and good availability and workability. After a detailed material selection process, described in \Chapref{sensorMaterialsApp}, wood, bamboo, carbon fibre reinforced polymers (CFRP) and glass fibre reinforced polymers (GFRP) were found to be the most appropriate materials. 

While bamboo is the best material to use for the sensor mount frame based on it's vibrational characteristics and stiffness alone, it is difficult to obtain in Australia in structural form. Wood is the next best option, being very easy to work with and readily available. GFRP has a similar stiffness to wood but with worse damping properties and a higher density. On the other hand, CFRP has a much better stiffness however it sacrifices damping properties and density. The two composites are also more difficult to work with, and are more expensive. Hence, the best material to use for the sensor mount is wood. In order to meet the stiffness requirements, a structural grade timber such as MGP10 will be used. 
 
\nomenclature{CFRP}{Carbon Fibre Reinforced Polymer}%
\nomenclature{GFRP}{Glass Fibre Reinforced Polymer}%

\subsection{Steering focal point}
The sensors need to be mounted in a configuration that will keep individual sensor elements aligned, so that corresponding ground information can be collated and analysed correctly. In the case of a sensor array, the mounting arrangement must also provide good sensor coverage, and ensure that the platform does not travel over unscanned area. In order to determine the best mounting location for the sensor mount, simulations were performed for front wheel and rear wheel steering (\Chapref{steeringSelection}). The results, presented in \Figref{turnCoverage}, show that rear wheel steering provides a tighter turning radius and better sensor coverage. Hence, the sensors will be mounted to the rear of the quad bike, which will be driven in reverse to provide real wheel steering capabilites. 
\begin{figure}[ht]
\centerline{
\begin{tabular}{ccc}
\subfloat[Front wheel steering]{\includegraphics[width=0.4\textwidth]{3-ConceptDesign/Detector_Coverage1.png}}
& \subfloat[Rear wheel steering]{\includegraphics[width=0.4\textwidth]{3-ConceptDesign/Detector_Coverage2.png}}
\end{tabular}}
\caption{Sensor coverage simulations} 
\figlabel{turnCoverage}
\end{figure}

\subsection{Initial designs}
Knowing the basic sensor requirements, the selected material for the mount, and the mounting location on the quad bike, preliminary designs for sensor mount were developed. 

There are three support points at the rear of the quad bike to which the mount can be attached, a tow hitch near the rear axle, and two metal brackets beneath the rear tray. The tow hitch has a level surface and is capable of supporting a vertical load of 14 kg, which means that a cantilever beam could be mounted on the tow hitch, but its weight would be limited. To fix this, diagonal support braces can be used, attached to the two metal brackets. Based on this geometry, a basic design for the sensor mount is shown in \Figref{basic}.

\begin{figure}[ht]
\includegraphics[width=.65\textwidth]{3-ConceptDesign/basic.PNG}
\centering
\caption[Basic sensor mount design]{Basic sensor mount design showing central wooden beam and metal supports}
\figlabel{basic}
\end{figure}

A multi-beam structure is used rather than a single cantilever in order to reduce mass and improve torsional rigidity. The main beam rests on the tow hitch, and is made of a structural grade timber. The two diagonal supports could also be made of wood, however a metal such as mild steel provides a more rigid frame. The total length of the frame is 1.5 m, which allows enough length for the metal detector and GPR to be mounted with the required minimum proximity. 

The GPR came with a small attachment which could be fixed to the antenna heads (\Figref{GPRmount} (a)). This attachment had a threaded head, allowing it to screw into a standard diameter broom or mop handle. In order to operate the GPR as close to the ground as possible, it was decided that a telescopic arm should be used, which could be used to maintain contact between the GPR head and the ground at all times. To prevent damage from occurring to the GPR in case it hit an obstacle, the arm would be installed on a rotating section.

\begin{figure}[ht]
\centerline{
\begin{tabular}{cc}
\subfloat[GPR attachment]{\includegraphics[height=.35\textwidth]{3-ConceptDesign/GPRmount2.jpg}}
& \subfloat[Telescopic arm and GPR attachment]{\includegraphics[height=.35\textwidth]{3-ConceptDesign/GPRmount.PNG}}
\end{tabular}}
\caption{Attachment of the GPR to the sensor mount}
\figlabel{GPRmount}
\end{figure}

It was decided to encase the metal detector between two sheets of Perspex, since the foam body of the metal detector was susceptible to damage. This also allowed the metal detector to be mounted more easily, since holes could be put through the Perspex. The metal detector assembly is shown in \Figref{MD}.

\begin{figure}[ht]
\includegraphics[width=.45\textwidth]{3-ConceptDesign/MD.PNG}
\centering
\caption{Metal detector with Perspex covers}
\figlabel{MD}
\end{figure}

There were several design requirements for the attachment of the metal detector panel, which are listed below. 

\begin{itemize}
\item Ease of manufacture: As with the GPR attachment, the attachment for the metal detector must be simple to manufacture in a short time frame. A design with fewer components and simpler joins is preferred.
\item Strength: The attachment must be able to support the 6 kg weight of the metal detector assembly. 
\item Rigidity: The attachment must to rigid enough to prevent the metal detector from vibrating excessively during operation.   
\item Ground clearance: The chosen design must be able to maintain a low ground clearance. Preference is given to a design in which this height can be varied to suit the terrain conditions. 
\item Aesthetics: The design must be aesthetically pleasing. 
\end{itemize}

Four different design concepts were developed, shown in \Figref{MDdesigns}.
\begin{figure}[ht]
\centerline{
\begin{tabular}{cc}
\subfloat[Design 1]{\includegraphics[height=0.35\textwidth]{3-ConceptDesign/square.PNG}}
& \subfloat[Design 2]{\includegraphics[height=0.35\textwidth]{3-ConceptDesign/final.PNG}}\\
\subfloat[Design 3]{\includegraphics[height=0.35\textwidth]{3-ConceptDesign/triangle.PNG}}
& \subfloat[Design 4]{\includegraphics[height=0.35\textwidth]{3-ConceptDesign/cantilever.PNG}}
\end{tabular}}
\caption{Concepts for attaching the metal detector the the sensor mount} 
\figlabel{MDdesigns}
\end{figure}

Design 1 has four simple vertical supports, which can be fixed to the Persex sheets of the metal detector assembly. The wooden supports are connected to the body of the sensor mount through lap joints, providing strength and rigidity. Design 2 is similar to Design 1, however nylon rods are used to support the metal detector panel instead. This allows the height of the metal detector to be varied to suit the terrain conditions. Furthermore, if the sensor mount were to encounter an obstacle, the nylon rod would shear before damage occurred to the metal detector. Vertical supports are added to provide torsional stiffness, and nylon bolts and wooden cross pieces are used to hold the rods firmly in place. Design 3 is similar to Design 1, with diagonal supports connected to a flat end section. Design 4 has a cantilever-like structure, with a single cross beam used to attach to metal detector assembly at two points. All four designs were evaluated against the design requirements, the results of which are shown in \Tabref{MDmatrix}. Consequently, Design 2 was chosen for the final sensor mount concept, shown in \Figref{finalSensorFrameDesign}. 

\begin{table}[ht]
\centering
\caption{Selection of metal detector attachment design}
\tablabel{MDmatrix}
\begin{tabular}{r *5c}
    \multicolumn{1}{r}{}  & \mcrot{1}{l}{30}{Design 1} & \mcrot{1}{l}{30}{Design 2} & \mcrot{1}{l}{30}{Design 3} & \mcrot{1}{l}{30}{Design 4}\\ \toprule 
    Ease of manufacture & 3 & 3 & 2 & 4\\ 
    Strength & 5 & 5 & 4 & 2\\ 
    Rigidity & 5 & 4 & 4 & 2\\ 
    Ground clearance & 2 & 5 & 2 & 2\\ 
    Aesthetics & 3 & 3 & 4 & 3\\  \midrule
    \textbf{Total} & 72\% & 80\% & 64\% & 52\%\\ \bottomrule
\end{tabular}
\end{table}

\begin{figure}[ht]
\includegraphics[width=0.55\textwidth]{3-ConceptDesign/finalDesign.PNG}
\centering
\caption{Final sensor mount concept with metal detector and GPR attached} \figlabel{finalSensorFrameDesign}
\end{figure} 


\section{Subsystem integration}
Suitable consideration must be made to ensure that the individually developed systems are capable of integrating effectively to form a single continuous system. At the concept stage of the design the interacting points at the boundary limits of these systems were identified. These include the boundaries between physical systems, and the virtual boundaries between software and electrical systems. Early identification of these interactions ensured that the integration components necessary were considered as part of the detailed design.

\seclabel{conceptprojectdesign}

% \subsection{Electronics}
% As a major component of the project is advanced signal processing in real time, significant computational capabilities will be required on the quad bike. In addition to signal processing, other platform software systems such as vehicle control and telemetry will also be running. To achieve this without requiring multiple discrete hardware components (which would require communications input/output (I/O) interfaces to share data), a single hardware system capable of executing multiple threads simultaneously and asynchronously is required on the vehicle. The hardware system executing the signal processing software must also be capable of reading sensory input from USB devices, as this is the communications format available on the SIRO-PULSE II GPR unit. 
% \nomenclature[A]{I/O}{Input/Output}% 

% A second major component of the project is the automation of the quad bike, requiring software control over a series of actuators and sensors. To provide the greatest fidelity of control over the vehicle, the electronics hardware used to interface with the actuators and sensors must be capable of reading and writing to low-level I/O devices quickly and with minimal latency or overhead. 

% \begin{itemize}
% \item \textbf{Bespoke Electronics}\\
% Bespoke electronic equipment has the capacity to allow incredibly fast access to I/O devices through the use of task-specific commercial off-the-shelf (COTS) chips. However, the time consumption and expense of planning an entirely hardware-driven control system for anything more than trivial data handling is inappropriate for this project. The inability to prototype as with software means that the ability to test and then revisit a solution is not possible, and a hardware/purely electronics driven system is not capable of general purpose processing. Therefore, this is not a realistic option for achieving the project aims.
% \nomenclature[A]{COTS}{Commercial off-the-shelf}% 
% \item \textbf{Microcontrollers}\\
% Microcontrollers have become the de facto standard for small to medium software-based projects which require access to physical sensors and actuators, due to their readily available access to low level I/O. Microcontrollers supporting common languages such as C++ and Java allow easy development and rapid prototyping, though the inability to easily connect debugging equipment or generate test output slows the development process. Microcontrollers are inexpensive and provide high I/O availability but at the cost of limited processing power. The low-level nature of microcontrollers means that desirable features like hardware interrupts are exposed to and accessible by developers. 
% \item \textbf{Desktop computing equipment/Laptop} \\
% Conventional desktop computing equipment is the fastest general purpose computing hardware that will be available to the project. In addition to having the greatest computing power, it has the highest ability to support prototyping and allows for rapid software development with readily accessible software generation and debugging tools. The drawback of this higher-level computing platform is the reduced accessibility of low-level I/O devices, and the amount of computing overhead caused by operating system processes. Operations that require fast I/O access may be hampered by the inability to ensure thread availability, and so for robust operation this may require buffering to a secondary, lower level device.
% \end{itemize}

% None of the individual items presented allow for the full range of requirements of this project. As a result, the general concept for the hardware arrangement to execute the software systems is shown below in \Figref{hardwareLayout}.
% % is this figure text fucking big enough maziar??
% \begin{figure}[ht]
% \includegraphics[width = \textwidth]{3-ConceptDesign/electronics.png}
% \centering
% \caption{Conceptual hardware layout} \figlabel{hardwareLayout}
% \end{figure}

% Under this system, the project will use standard desktop computing equipment for the bulk of the software, to make use of its superior processing power and the rapid development it allows. This device will be the data handler and processor, and act as the 'central' software location for the project. Sensors and actuators that require low level I/O access will be connected to a secondary microcontroller, which will act independently to buffer inputs and outputs of the system, which can then be communicated to the primary computer over a serial communications connection. The project will not aim to develop any custom electronics boards and handle all signal amplification or processing in software.

\subsection{Software architecture}
The autonomous quad bike will be used as the platform on which the sensor mount and software subsytems will operate. To achieve the framework for overall automation of the platform the integration of the actuator electronics, navigation, sensors, signal processing and operator device is required. This framework is completed through the Central Hub as shown in \Figref{central} where dotted lines indicate implementations of the interfaces. The Central Hub will combine and read all the relevant processes and transmit the required information to the operator device. This enables all processes to be completed in parallel on independent threads, satisfying the deliverables as defined in \secref{primary}.

\begin{figure}[ht]
\includegraphics[width=\textwidth]{3-ConceptDesign/fyp_structure.png}
\centering
\caption{Integration of automation, navigation, sensors and signal processing} 
\figlabel{central}
\end{figure}

The landmine detection system as a whole requires at minimum two control devices: the control system on the remote platform, and the control device held by a remote operator. The various subsystems involved in the software design were considered when selecting these control devices, and the decision was made to place the majority of software base on the on the remote platform. Having the majority of software systems physically residing on the remote platform allows the vehicle to remain in control of all systems in the event of a communications loss, meaning that necessary actions to stop the vehicle can be made if a severance of communications is detected. Under this scheme, the operator device would act only as a remote terminal to relay data to the central processing system.

\subsection{Electronics}
A conventional PC running a Windows XP operating system was chosen for the primary control device on the remote platform, as this was a requirement to interface with the supplied GPR and metal detector. The PC also allowed for maximum ease of development under a familiar operating environmental, and provided tools for fast debugging of code.

The control device on the remote platform needed to provide low level I/O to support the sensors and actuators controlling the motion of the quad bike. This functionality is available through a conventional PC, however can be difficult to configure correctly and suffers from delay and buffering issues caused by thread switching. The options available were to retain the PC as the sole electronic controller, or to push the control of low level sensors and actuators to a microcontroller which was better suited to the task. The microcontroller was selected due to its low cost, high I/O flexibility and the ability to manage I/O at a consistent cycle rate. 


\subsection{Communications}
Communications between electronics systems inevitably introduces latency in signals and so the relationships between electronics systems were chosen to minimise the amount of data that was required to be transmitted. This also minimises the possibility of the communications channel becoming saturated and blocking the transmission of important time critical signals, such as an emergency stop. Where possible, data will be retained within the bounds of a single controller and only transferred when absolutely required. The major communications interfaces in the control system are detailed in the diagram below.

\begin{figure}[ht]
\includegraphics[width=0.75\textwidth]{3-ConceptDesign/communications.png}
\centering
\caption{Integration of automation, navigation, sensors and signal processing} 
\figlabel{communications}
\end{figure}

An example of our minimal communications strategy is evident from the above figure. The output from the GPS unit is a series of registers which is reported to the Arduino at a rate of approximately 10 Hz. Some degree of processing is required to translate this block of data into a geographic position, which can be represented by only two numbers. To reduce communications overhead, the Arduino is responsible for performing this calculation and will only report the position to the PC when explicitly requested to. This limits the volume of data required to be transmitted, and can limit the rate at which the transmission is sent. As the PC is the central unit for all other systems, the PC also maintains control over all of the communications interfaces and is responsible for initiating all communication traffic.

\end{document}

















