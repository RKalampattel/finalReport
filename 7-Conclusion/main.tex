\documentclass[main.tex]{subfiles}

\begin{document}
\chapter{Conclusion and Future Work}
\chaplabel{conclusion}
This chapter discusses to what extent each of the project objectives has been achieved. In addition, recommendations are made for future work, and a conclusion is presented.  

\textcolor{red}{\textbf{Remember to update the objectives below if any changes are made to them in the introduction}}

\section{Outcomes}
The outcomes for the project are assessed against the primary and extended objectives, as outlined in \secref{projectGoal}.

\subsection{Primary objectives}
\begin{enumerate}
\item \textbf{Selection and subsequent modification of an existing platform for remote control via a handheld device}\\ 
A quad bike supplied by the DTSG was chosen as the platform for the project. 
After an evaluation of the existing hardware, the subsystems to be modified were identified. 
A new power supply was installed for the steering motor, while a linear transducer was installed to more accurately measure the position of the brake actuator. 
The result of this was improved reliability in positioning of both the steering and braking systems, culminating in a controllable platform capable of being automated.
New linear potentiometers were also installed for the gear selector to replace damaged units, returning functionality to this system. 
The DragonBoard control unit was replaced with a more modern Arduino microcontroller, and the entire system was rewired. 
This provided the capability for connection of the newer GPS and inertial sensors, and allowed more rapid development of the new control software.
All subsystems were tested independently, and after the development of the tablet application simultaneous tests were successfully conducted resulting in full remote controllability.

 
% Necessary systems for control, such as throttle, steering, gear and brake actuation, and communication will be implemented on the platform. Each system will be tested separately to ensure that it functions correctly, meets the project requirements, and is safe to operate remotely. Once all systems are operating independently, simultaneous operation shall be conducted with the aim to maintain controllability of the platform at all instances. Successful completion of this objective, once the platform is able to be operated remotely, is the first step towards full platform automation.   

\item \textbf{Development of software for platform automation and basic navigation, allowing for the platform to travel autonomously under supervision from an operator}\\ 
After conducting a literature review, different methods for navigation were identified. Eventually, the pure pursuit method was selected for path tracking as it provided the best performance at low speeds and discontinuous curvatures, which were expected to be encountered. An Extended Kalman Filter was used to create a highly accurate positioning system, overcoming the limitations of a standalone GPS or IMU. After modification of the platform and the upgrade of control unit were completed, code for automation was developed and integrated intro the navigation software. A virtual platform was developed, enabling testing of the navigation and automation software in a controlled virtual environment. 

The resulting system was a navigation system that could successfully follow a defined path, with minimal deviation necessary only to complete sharp turns. The virtual platform correctly simulated the response of the vehicle to the inputs generated, and the system demonstrated a degree of robustness against sensory positional error thanks to the introduction of the Kalman filter. Testing of the software with the quad bike was also completed, leading to the tuning of the automation code and of the virtual platform. This work is ongoing.
% Initially, automation and navigation software will be tested in a virtual environment, allowing for full simulation of pre-defined scenarios in a risk-free manner. The outputs of this virtual platform will include the location, heading, and speed of the platform. On completion of objective 1, the software and control systems will be integrated into the real platform, and further testing will be performed. After completing this objective, the platform will be fully autonomous.

\item \textbf{Automatic traversal of a user defined path or region}\\ 
The developed software allows the operator to specify the path of the quad bike in two ways, either by selecting waypoints which define a path, or by selecting points which define the boundaries of a region. If the latter is chosen, the navigation software subdivides the region and generates a path to scan the area systematically. 
This delivered system is intuitive for an operator to use and allows for a path to be defined quickly. The tablet device also allows the operator to visualise the path the system will attempt to take on the map view, allowing for changes to be made if necessary. All features of the previous requirement work as expected when using user-defined paths. 
% In real world scenarios, it is important for the operator to be able to specify the path of the platform as they wish. Initially, single waypoint navigation will be implemented, allowing an operator to instruct the platform to travel from one predefined point to another before coming to a stop. Once this is achieved, the operator will be able to define waypoints that enclose a region. The platform will then subdivide the region and generate its own path to traverse. Software will again be tested in a virtual environment before being implemented into the real platform. Completion of this objective will extend the navigational capabilities of the platform, and make it easier for an operator to define a mission. 

\item \textbf{Modification of an existing platform to accommodate mine detection sensors}\\ 
A sensor mount was designed based on the requirements of the available sensing equipment, and the constraints imposed by the platform. The minimum proximity of the metal detector and GPR dictated the length of the mount, while the distance to the ground for both sensors was kept as small as possible to maximise the signal to noise ratio. A structural and vibrational analysis was performed to ensure that the mount would be able to support the desired loads while isolating the sensors from the vibrations of the quad bike. The body of the mount was made from structural pine, while the supports were made from steel. Nylon rods were used to hold the metal detector in place, providing an interference free environment for the sensor while allowing for its height to be adjusted.   
Testing has shown that the detection capacity of the individual sensors is unhindered by attachment to the quad bike, under both stationary and engine idling scenarios, confirming the results of the vibrational analysis performed on the mount design.
% A sensor mount will be designed and built so that it can attach to the platform and support the chosen sensors safely, while adhering to the operating requirements of both the platform and sensors. The most suitable design will be chosen from several concepts based on its ability to isolate the sensor suite and minimise sources of interference to the sensor output. On completion of this objective, the platform will be fully integrated with the sensor suite.

\item \textbf{Development of software for detection and autonomous classification of subsurface objects, with a focus on reducing the number of incorrect identifications or false positives}\\ 
Initial tests with sensors were used to identify metrics that could characterise landmines. Based on the findings of the literature review, algorithms were developed to find these metrics from sensor data in real time. Extensive tests were performed using different dummy landmines and clutter objects in different soil types, which led to the development of a database of metrics. A sensor fusion algorithm was also developed, which allowed a final decision to made regarding an identified object, and presented a confidence interval on the likelihood of the object being a landmine. Testing of the sensor system in real-time while mounted on the quad bike remains to be completed.
% After initial testing, the output signals from the metal detector and GPR will be processed in real time. The results will be analysed with the aim of identifying suitable metrics that can be used to classify and differentiate various targets. A database of metrics will produced and used to identify the likelihood of a landmine being present. Operational trials will be conducted to test the effectiveness of the developed system. On completion of this objective, the sensor suite will be able to detect targets in real time and return a supporting confidence interval.
\end{enumerate}

\subsection{Extended objectives}
\begin{enumerate}
\item \textbf{Development of a tablet application that allows for communication with the platform and sensor systems}\\ 
A tablet application has been developed for an android system. The primary focus during development of this system has been on maximising the ease-of-use of this device, which has resulted in an interface with large, intuitive buttons and only a few essential controls to . 

% Rather than receiving information on a computer, a portable handheld device such as a tablet can be used. A dedicated application will be developed, allowing for even an unskilled operator to carry out a complex mission. The operator will be able to control and monitor the progress of the platform in real time, and once an object is detected by the sensor system, the operator will also be notified. A GPS coordinate will be assigned to the object, and based on the returned confidence interval, the operator can make an informed decision about how to proceed.


\item \textbf{Provision of a live video feed to the operator}\\ 
A camera has been purchased and installed on the quad bike, enabling a live video feed to be seen by operator, however this is currently only possible using a third party application, not the tablet application.
This is due to constraints on the codecs that are capable of being decoded for live video streams using the native android system. The remote tablet application currently supports viewing of live video streams that are H.264 AVC compliant, however the video codec used by the purchased video camera is not entirely compliant with this codec specification, contrary to the claims made on the packaging which affirm full compliance.
% A live video feed allows for an operator to have a better understanding of the operating conditions. A camera will be installed on the platform, and the output will be integrated with the developed tablet application.  This will allow the operator to detect and respond appropriately to unknown obstacles in the search area. 

\item \textbf{Physical marking of landmine locations by the platform at the time of detection}\\ 
Due to time constraints, options for physical marking systems were not explored. The current state of the quad bike and the sensor mount support the addition of further equipment which could be used for this purpose, and there is capacity in the electronic system for attachment of additional actuators, meaning this is something that could be attempted in future.  
% GPS coordinates alone may not provide sufficient precision for identifying the location of landmines. An electromechanical system that physically marks the location would allow demining personnel to identify this location more clearly.
\end{enumerate}

\section{Future work}
There are several ways in which the project could be extended if it were to be continued in future. 

% Incomplete objectives
As mentioned in the previous section, several objectives are yet to be completed. Firstly, full integration of the sensor and platform needs to be done, after which detected objects can be logged. Physical marking of landmines and the addition a live feed to the tablet application are also possible tasks.

% Improvements to things we have done already
Improvements that can be made: more testing for sensors, brake actuator, wheel encoder

% New additions
Things that were outside of the current scope such as object detection (more complicated navigation), use arrays instead of line scan sensors, 

\section{Conclusion}
The authors have developed an autonomous platform for landmine detection, and successfully demonstrated its feasibility. The challenges associated with this process, and the methods used to overcome these challenges, have been outlined in this report. It is hoped that this information will be beneficial in developing the next generation of autonomous landmine detection platforms.  

\end{document}