\documentclass[main.tex]{subfiles}

\begin{document}
\chapter{Conclusion and Future Work}
\chaplabel{conclusion}
\textcolor{red}{Intro for this chapter. No new information to be presented!}

\section{Outcomes}
\textcolor{red}{List the project objectives, outline achievements for each based on the measure of completion}

\subsection{Primary objectives}
\begin{enumerate}
\item Selection and subsequent modification of an existing platform for remote control via a handheld device. Necessary systems for control, such as throttle, steering, gear and brake actuation, and communication will be implemented on the platform. Each system will be tested separately to ensure that it functions correctly and is safe to operate remotely. Once all systems are operating independently, simultaneous operation shall be conducted with the aim to maintain controllability of the platform at all instances. Successful completion of this objective, once the platform is able to be operated remotely, is the first step towards full platform automation.  

\item Modification of an existing platform to accommodate a metal detector and a GPR. A sensor mount will be designed and built so that it can attach to the platform and support the chosen sensors safely, while adhering to the operating requirements of both the platform and sensors. The most suitable design will be chosen from several concepts based on its ability to isolate the sensor suite and minimise sources of interference to the sensor output. On completion of this objective, the platform will be fully integrated with the sensor suite.

\item Development of software for platform automation and navigation, allowing for the platform to travel autonomously under supervision from an operator. Initially, automation and navigation software will be tested in a virtual environment, allowing for full simulation of operations in a risk-free manner. The outputs of this virtual platform will include the location, heading, and speed of the platform. On completion of objective 1,  the software and control systems will be integrated into the real platform, and further testing will be performed. After completing this objective, the platform will be fully autonomous.

\item Development of software for detection and autonomous classification of subsurface objects, with a focus on reducing the number of incorrect identifications or false positives. After initial testing, the output signals from the metal detector and GPR will be processed in real time. The results will be analysed with the aim of identifying suitable metrics that can be used to classify and differentiate various targets. A database of metrics will produced and used to identify the likelihood of a landmine being present. Operational trials will be conducted to test the effectiveness of the developed system. On completion of this objective, the sensor suite will be able to detect targets in real time.

\item Logging of detected objects with GPS coordinates. Once an object has been located, its position will be recorded and transmitted to a handheld device used by an operator. The output from the detection software in objective 4 will also be made available, allowing the operator to make an informed decision about how to proceed. Once the sensor suite has integrated with the platform (objective 2) and the platform is fully autonomous (objectives 1 and 3), this is the final objective required to complete the project goal.
\end{enumerate}

\subsection{Extended objectives}

\begin{enumerate}
\item Automatic traversal of a designated area enclosed by a user defined region. Initially, single waypoint navigation will be conducted, requiring the platform to travel from one predefined point to another before coming to a stop. In this objective, waypoints enclosing an area will be used, requiring the platform to generate its own path to traverse the enclosed area. Software will be tested in a virtual environment before being implemented into the platform on completion of the primary objectives. Completion of this objective will extend the navigational capabilities of the platform, and make it easier for an operator to use. 

\item Provision of a live video feed as well as live updates of vehicle location and direction to the operator's handheld device. While the operator will be able to view the location of the platform using a handheld device, a live video feed allows for greater visualisation of operating conditions. This would also provide a greater ability to detect and respond appropriately to unknown obstacles in the search area.

\item Physical marking of landmine locations by the platform at the time of detection. GPS coordinates alone may not provide sufficient precision for identifying the location of landmines. An electromechanical system that physically marks the location would allow demining personnel to identify this location more clearly.
\end{enumerate}

\section{Future work}
\textcolor{red}{Any uncompleted work (extended objectives), things identified during the project that could need some more work if the project were to continue next year (e.g. faster brake actuator), and more general stuff}

\section{Conclusion}


\end{document}